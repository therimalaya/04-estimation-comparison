%review=doublespace preprint=single 5p=2 column
\documentclass[12pt,3p,authoryear]{elsarticle}

%% add packages %%
%% ------------ %%
\usepackage[hyphens]{url}
\usepackage{graphicx}
\usepackage{booktabs}
\usepackage[T1]{fontenc}
\usepackage{lmodern}
\usepackage{caption}
\usepackage{subfig}
\usepackage{amssymb, amsmath}
\usepackage[inline]{enumitem}
\usepackage{float}
\usepackage{tabularx}
\usepackage[dvipsnames, table]{xcolor}
\usepackage{ifxetex, ifluatex}
\usepackage{fixltx2e}
\usepackage[unicode=true, colorlinks]{hyperref}
\usepackage{cleveref}
\usepackage{tabu}
\usepackage{mathpazo}
%% ------------ %%

%% Conditional Packages %%
%% -------------------- %%

\usepackage{easyReview}



% use upquote if available, for straight quotes in verbatim environments
\IfFileExists{upquote.sty}{\usepackage{upquote}}{}

\ifnum 0\ifxetex 1\fi\ifluatex 1\fi=0 % if pdftex
  \usepackage[utf8]{inputenc}


\else % if luatex or xelatex
  \usepackage{fontspec}
  \ifxetex
    \usepackage{xltxtra,xunicode}
  \fi
  \defaultfontfeatures{Mapping=tex-text,Scale=MatchLowercase}
  \newcommand{\euro}{€}



    \setmonofont{sourcecodepro}


\fi

% use microtype if available
\IfFileExists{microtype.sty}{\usepackage{microtype}}{}






\usepackage{longtable}




% Pandoc toggle for numbering sections (defaults to be off)
\setcounter{secnumdepth}{5}

%% Use Landscape Pages
\usepackage{lscape}

\usepackage{setspace}
\setstretch{1.5}

\usepackage{lmodern}

%% -------------------- %%

%% Create and Provide some customizations %%
%% -------------------------------------- %%
\providecommand{\tightlist}{%
  \setlength{\itemsep}{0pt}\setlength{\parskip}{0pt}}
  
%% Custom macros
\newtheorem{mydef}{Definition}
\newcommand{\bs}[1]{\ensuremath{\boldsymbol{#1}}}
\newcommand{\diag}[1]{\mathrm{diag}\left(#1\right)}
\newcommand{\seq}[3][1]{\ensuremath{#2_{#1},\ldots,\,#2_{#3}}}
\newcommand{\note}[1]{\marginpar{\scriptsize\tt{\color{RoyalBlue}#1}}}
\newcommand{\edit}[1]{{\color{OrangeRed} #1}}

%% Declare Operators
\newcommand{\argmin}{\operatornamewithlimits{arg\,min}}
\newcommand{\argmax}{\operatornamewithlimits{arg\,max}}

% set some lengths
\setlength{\parindent}{0pt}
% \setlength{\parskip}{6pt plus 2pt minus 1pt}
\setlength{\emergencystretch}{3em}  % prevent overfull lines

%% Hyperref color setup
\AtBeginDocument{%
  %% Define Colors
  \newcommand\myshade{80}
  \colorlet{mylinkcolor}{violet!\myshade!black}
  \colorlet{mycitecolor}{YellowOrange!\myshade!black}
  \colorlet{myurlcolor}{Aquamarine!\myshade!black}

  \hypersetup{
    breaklinks = true,
    bookmarks  = true,
    pdfauthor  = {},
    pdftitle   = {Comparison of Multivariate Estimation Methods},
    linkcolor  = mylinkcolor,
    citecolor  = mycitecolor,
    urlcolor   = myurlcolor,
    colorlinks = true,
  }
}
\urlstyle{same}  % don't use monospace font for urls
%% -------------------------------------- %%

%% Customizations %%
%% -------------- %%
 % turn line numbering on

%% -------------- %%

%% Configure Bibliography %%
%% ---------------------- %%
\bibliographystyle{elsarticle-harv}
\biboptions{numbers,sort&compress}

\makeatletter
\providecommand{\doi}[1]{%
  \begingroup
    \let\bibinfo\@secondoftwo
    \urlstyle{rm}%
    \href{http://dx.doi.org/#1}{%
      doi:\discretionary{}{}{}%
      \nolinkurl{#1}%
    }%
  \endgroup
}
\makeatother

% 

%% Header Includes %%
%% --------------- %%
%% --------------- %%



\begin{document}
%% --- Front Matter Start --- %%
\begin{frontmatter}

  \title{Comparison of Multivariate Estimation Methods}
  
    \author[KBM]{Raju Rimal\corref{c1}}
   \ead{raju.rimal@nmbu.no} 
   \cortext[c1]{Corresponding Author}
    \author[KBM]{Trygve Almøy}
   \ead{trygve.almoy@nmbu.no} 
  
    \author[NMBU]{Solve Sæbø}
   \ead{solve.sabo@nmbu.no} 
  
      \address[KBM]{Faculty of Chemistry and Bioinformatics, Norwegian University of Life
Sciences, Ås, Norway}
    \address[NMBU]{Prorector, Norwegian University of Life Sciences, Ås, Norway}
  
  \begin{abstract}
  Prediction performance often does not reflect the estimation behaviour
  of a method. High error in estimation not necessarily results in high
  prediction error but can leads to an unreliable prediction when test
  data are in a different direction than the training data. In addition,
  the effect of a variable becomes unstable and can not be interpreted in
  such situations. Many research fields are more interested in these
  estimates than performing prediction. This study compares some
  newly-developed (envelope) and well-established (PLS, PCR) prediction
  methods using simulated data with specifically designed properties such
  as multicollinearity, correlation between multiple responses and
  position of principal components of predictor that are relevant for the
  response. This study aims to give some insight on these methods and help
  researcher to understand and use them for further study. \emph{Write
  some specifics from the results to show what we have found.}
  \end{abstract}
   \begin{keyword} model-comparison,multi-response,simrel,estimation,estimation error,meta modeling\end{keyword}

\end{frontmatter}

\section{Introduction}\label{introduction}

Estimation of parameters in a regression model is an integral part in
many research study. Research fields such as social science,
econometric, psychology and medical study are more interested in
measuring the impact of certain indicator or variable rather than
performing prediction. Such studies has large influence in people's
perception and also help in policy making and decisions.

Technology has facilitated researcher to collected large amount of data
however often times, such data either contains irrelevant information or
are highly collinear. Researchers are devising new estimators to extract
information and identify their inter-relationship. Some estimators are
robust towards fixing multicollinearity problem while some are targeted
to model only the relevant information content in response variable.

This study extends the \citep{rimal2019pred} and compares some well
established estimators such as Principal Components Analysis (PCA),
Partial Least Squares (PLS) together with two new methods based on
envelope estimation: Envelope estimation in predictor space (Xenv)
\citep{cook2010envelope} and simultaneous estimation of envelope (Senv)
\citep{cook2015simultaneous}. The estimation process of these methods
are discussed in {[}Methods{]} section. The comparison tests the
estimation performance of these methods using multi-response simulated
data from linear model with controlled properties. The properties
includes the number of predictors, level of multicollinearity,
correlation between different response variables and the position of
relevant predictor components. These properties are explained in
{[}Experimental Design{]} section together with the strategy behind the
simulation and data model.


\renewcommand\refname{References}
\bibliography{ref-db.bib}


\end{document}
